\documentclass[12pt]{article}
\input{bayesuvius.sty}


\begin{document}
\title{Causal DAG Extraction from 3 Short Stories  \\
and 3 Movie Scripts}
\date{ \today}
\author{Robert R. Tucci\\
        tucci@ar-tiste.com}
\maketitle
\vskip2cm
\section*{Abstract}
We improve a previously proposed algorithm
for doing causal DEFT (DAG Extraction from Text), and then we apply the new algorithm to 2 usecases:
3 short stories by P.G. Wodehouse and 3 movie scripts by Pixar/Disney. The software  used to accomplish this 
endeavor is called ``Mappa Mundi" and is available as open source at GitHub. 
\newpage

In this paper, I improve an algorithm
for doing causal DEFT (DAG Extraction from Text)
that was
first proposed in Ref.\cite{deft1}
I then apply the new algorithm to 2 usecases:
\begin{enumerate}

\item 3 short stories by P.G. Wodehouse
(the text from these was obtained from the
Project Gutenberg website \cite{project-gutenberg}) 

\begin{itemize}
\item  Bill the Bloodhound
\item  Extricating Young Gussie
\item Wilton's Holiday
\end{itemize}

\item 3 movie scripts by Pixar/Disney.
(the text for these was obtained from the IMSDb website Ref.\cite{imsdb})\footnote{The Mappa Mundi repo at GitHub
contains a Python script called {\tt downloading.py}
that uses the BeautifulSoup 
package to scrape all the $1100+$ movie
scripts available at the IMSDb website.
My original intention
was to apply my algorithm to 
all of those movie scripts.
However, due to lack of
hardware resources, I had to 
settle for just 3 movie scripts.}

\begin{itemize}
\item Toy Story 
\item Up
\item WALL-E
\end{itemize}
\end{enumerate}

The Python script
software that was used to 
accomplish this endeavor is open source and available
at Github (see Ref.\cite{github-mappa-mundi})



\bibliographystyle{plain}
\bibliography{references}
\end{document}
